\documentclass[11pt,a4paper]{article}

\usepackage{amsmath}
\usepackage{amssymb}
\usepackage{bm}
\usepackage{graphicx}
\usepackage{algorithm}
\usepackage{algorithmic}
\usepackage{geometry}
\usepackage{hyperref}
\usepackage{booktabs}

\geometry{margin=1in}

\title{Model Predictive Control for Drone Obstacle Avoidance:\\Implementation Documentation}
\author{Drone Optimal Trajectory Project}
\date{\today}

\begin{document}

\maketitle

\tableofcontents
\newpage

\section{Introduction}

This document provides a comprehensive mathematical description of the Model Predictive Control (MPC) implementation for autonomous drone navigation with obstacle avoidance. The system combines gap-based navigation, artificial potential fields for reference trajectory generation, and optimization-based control with hard safety constraints.

\subsection{System Overview}

The control architecture consists of three main components:
\begin{enumerate}
    \item \textbf{Gap Navigator}: Analyzes LiDAR data to identify navigable corridors and compute intermediate waypoints
    \item \textbf{Reference Trajectory Generator}: Uses artificial potential fields to create smooth, obstacle-aware reference paths
    \item \textbf{MPC Solver}: Optimizes control inputs subject to dynamics, limits, and safety constraints
\end{enumerate}

\section{State and Control Variables}

\subsection{State Vector}

The full drone state at time step $k$ is represented as an 8-dimensional vector:
\begin{equation}
\bm{x}_k = \begin{bmatrix}
p_x \\ p_y \\ p_z \\ \psi \\ v_x \\ v_y \\ v_z \\ \dot{\psi}
\end{bmatrix} \in \mathbb{R}^8
\end{equation}

where:
\begin{itemize}
    \item $\bm{p} = [p_x, p_y, p_z]^T \in \mathbb{R}^3$: Position in world frame (ENU coordinates) [m]
    \item $\psi \in [-\pi, \pi]$: Yaw angle [rad]
    \item $\bm{v} = [v_x, v_y, v_z]^T \in \mathbb{R}^3$: Velocity in world frame [m/s]
    \item $\dot{\psi} \in \mathbb{R}$: Yaw rate [rad/s]
\end{itemize}

\subsection{Control Vector}

The control input at time step $k$ is:
\begin{equation}
\bm{u}_k = \begin{bmatrix}
a_x \\ a_y \\ a_z \\ \ddot{\psi}
\end{bmatrix} \in \mathbb{R}^4
\end{equation}

where:
\begin{itemize}
    \item $\bm{a} = [a_x, a_y, a_z]^T \in \mathbb{R}^3$: Linear acceleration in world frame [m/s$^2$]
    \item $\ddot{\psi} \in \mathbb{R}$: Yaw acceleration [rad/s$^2$]
\end{itemize}

\section{System Dynamics}

\subsection{Double Integrator Model}

The drone dynamics are modeled as a discrete-time double integrator system:

\subsubsection{Position Dynamics}
\begin{equation}
\bm{p}_{k+1} = \bm{p}_k + \bm{v}_k \Delta t
\end{equation}

\subsubsection{Velocity Dynamics}
\begin{equation}
\bm{v}_{k+1} = \bm{v}_k + \bm{a}_k \Delta t
\end{equation}

\subsubsection{Yaw Dynamics}
\begin{equation}
\psi_{k+1} = \psi_k + \dot{\psi}_k \Delta t
\end{equation}

\begin{equation}
\dot{\psi}_{k+1} = \dot{\psi}_k + \ddot{\psi}_k \Delta t
\end{equation}

where $\Delta t$ is the sampling time (typically 0.1 s).

\subsection{Compact Form}

The full system dynamics can be written as:
\begin{equation}
\bm{x}_{k+1} = \bm{f}(\bm{x}_k, \bm{u}_k) = \bm{A}\bm{x}_k + \bm{B}\bm{u}_k
\end{equation}

where $\bm{A} \in \mathbb{R}^{8 \times 8}$ and $\bm{B} \in \mathbb{R}^{8 \times 4}$ are:

\begin{equation}
\bm{A} = \begin{bmatrix}
\bm{I}_3 & \bm{0}_{3\times1} & \Delta t \cdot \bm{I}_3 & \bm{0}_{3\times1} \\
\bm{0}_{1\times3} & 1 & \bm{0}_{1\times3} & \Delta t \\
\bm{0}_{3\times3} & \bm{0}_{3\times1} & \bm{I}_3 & \bm{0}_{3\times1} \\
\bm{0}_{1\times3} & 0 & \bm{0}_{1\times3} & 1
\end{bmatrix}, \quad
\bm{B} = \begin{bmatrix}
\bm{0}_{3\times3} & \bm{0}_{3\times1} \\
\bm{0}_{1\times3} & 0 \\
\Delta t \cdot \bm{I}_3 & \bm{0}_{3\times1} \\
\bm{0}_{1\times3} & \Delta t
\end{bmatrix}
\end{equation}

\section{MPC Optimization Problem}

\subsection{Prediction Horizon}

The MPC optimizes over a finite prediction horizon:
\begin{itemize}
    \item $N$: Number of prediction steps (typically 20)
    \item $T = N \cdot \Delta t$: Total prediction time (typically 2.0 s)
\end{itemize}

\subsection{Cost Function}

The total cost function to minimize is:
\begin{equation}
J = J_{\text{terminal}} + \sum_{k=0}^{N-1} \left( J_{\text{tracking}} + J_{\text{goal}} + J_{\text{control}} + J_{\text{safety}} + J_{\text{obstacles}} \right)
\end{equation}

\subsubsection{Terminal Cost}

Penalizes deviation from the goal at the final prediction step:
\begin{equation}
J_{\text{terminal}} = Q_{\text{terminal}} \|\bm{p}_N - \bm{p}_{\text{goal}}\|_2^2
\end{equation}

Typical value: $Q_{\text{terminal}} = 150.0$

\subsubsection{Position Tracking Cost}

Penalizes deviation from the reference trajectory:
\begin{equation}
J_{\text{tracking}} = Q_{\text{pos}} \|\bm{p}_k - \bm{p}_{\text{ref},k}\|_2^2 + Q_{\text{yaw}} (\psi_k - \psi_{\text{ref},k})^2
\end{equation}

Typical values: $Q_{\text{pos}} = 15.0$, $Q_{\text{yaw}} = 2.0$

\subsubsection{Goal Attraction Cost}

Directly attracts the drone toward the goal:
\begin{equation}
J_{\text{goal}} = Q_{\text{goal}} \|\bm{p}_k - \bm{p}_{\text{goal}}\|_2^2
\end{equation}

Typical value: $Q_{\text{goal}} = 80.0$

\subsubsection{Control Effort Cost}

Penalizes large control inputs and encourages smooth control:
\begin{equation}
J_{\text{control}} = Q_{\text{vel}} \|\bm{v}_k\|_2^2 + R_{\text{acc}} \|\bm{a}_k\|_2^2 + R_{\text{yaw}} \ddot{\psi}_k^2 + R_{\text{jerk}} \|\bm{a}_k - \bm{a}_{k-1}\|_2^2
\end{equation}

Typical values: $Q_{\text{vel}} = 1.0$, $R_{\text{acc}} = 0.3$, $R_{\text{yaw}} = 0.5$, $R_{\text{jerk}} = 0.3$

\subsubsection{Velocity-Toward-Obstacles Penalty}

Penalizes velocities directed toward obstacles:
\begin{equation}
J_{\text{safety}} = Q_{\text{vel-obs}} \sum_{i=1}^{M_k} \max\left(0, -\frac{\bm{v}_k \cdot (\bm{p}_k - \bm{o}_i)}{\|\bm{p}_k - \bm{o}_i\|}\right)^2
\end{equation}

where $\bm{o}_i$ is the position of obstacle $i$, and $M_k$ is the number of obstacles at step $k$.

Typical value: $Q_{\text{vel-obs}} = 50.0$

\subsubsection{Repulsive Potential Field Cost}

Provides smooth obstacle avoidance through repulsive forces:
\begin{equation}
J_{\text{obstacles}} = Q_{\text{rep}} \sum_{i=1}^{M_k} \phi(\|\bm{p}_k - \bm{o}_i\|)
\end{equation}

The potential function $\phi(d)$ is defined as:
\begin{equation}
\phi(d) = \begin{cases}
\frac{1}{d - d_{\text{safe}}/2 + 0.5} \cdot \frac{d_0 - d}{d_0} & \text{if } d < d_0 \\
0 & \text{otherwise}
\end{cases}
\end{equation}

where:
\begin{itemize}
    \item $d = \|\bm{p}_k - \bm{o}_i\|$: Distance to obstacle $i$
    \item $d_{\text{safe}} = 0.8$ m: Safety radius
    \item $d_0 = 3.0$ m: Influence distance
\end{itemize}

Typical value: $Q_{\text{rep}} = 300.0$

\subsection{Constraints}

\subsubsection{Dynamics Constraints}

For all $k = 0, \ldots, N-1$:
\begin{equation}
\bm{x}_{k+1} = \bm{A}\bm{x}_k + \bm{B}\bm{u}_k
\end{equation}

\subsubsection{Initial Condition}
\begin{equation}
\bm{x}_0 = \bm{x}_{\text{current}}
\end{equation}

\subsubsection{Velocity Limits}

For all $k = 0, \ldots, N$:
\begin{equation}
\|\bm{v}_k\|_2 \leq v_{\max}
\end{equation}

Typical value: $v_{\max} = 2.0$ m/s

\subsubsection{Acceleration Limits}

For all $k = 0, \ldots, N-1$:
\begin{equation}
\|\bm{a}_k\|_2 \leq a_{\max}
\end{equation}

Typical value: $a_{\max} = 3.0$ m/s$^2$

\subsubsection{Yaw Rate Limits}

For all $k = 0, \ldots, N$:
\begin{equation}
|\dot{\psi}_k| \leq \dot{\psi}_{\max}
\end{equation}

Typical value: $\dot{\psi}_{\max} = 1.5$ rad/s

\subsubsection{Yaw Acceleration Limits}

For all $k = 0, \ldots, N-1$:
\begin{equation}
|\ddot{\psi}_k| \leq \ddot{\psi}_{\max}
\end{equation}

Typical value: $\ddot{\psi}_{\max} = 2.0$ rad/s$^2$

\subsubsection{Altitude Limits}

For all $k = 0, \ldots, N$:
\begin{equation}
z_{\min} \leq p_{z,k} \leq z_{\max}
\end{equation}

Typical values: $z_{\min} = 0.3$ m, $z_{\max} = 10.0$ m

\subsubsection{Obstacle Avoidance Constraints (Hard Constraints)}

For all $k = 0, \ldots, N$ and all obstacles $i = 1, \ldots, M_k$:
\begin{equation}
\|\bm{p}_k - \bm{o}_i\|_2 \geq d_{\text{safe}} - \epsilon_i
\end{equation}

where $\epsilon_i \geq 0$ are slack variables with penalty:
\begin{equation}
J_{\text{slack}} = W_{\text{obs}} \sum_{i=1}^{M_k} \epsilon_i^2
\end{equation}

Typical values: $d_{\text{safe}} = 0.8$ m, $W_{\text{obs}} = 5000.0$

\subsection{Complete Optimization Problem}

The complete MPC problem is formulated as:

\begin{equation}
\begin{aligned}
\min_{\substack{\bm{x}_{0:N}, \bm{u}_{0:N-1} \\ \bm{\epsilon}}} \quad & J_{\text{terminal}} + \sum_{k=0}^{N-1} \left( J_{\text{tracking}} + J_{\text{goal}} + J_{\text{control}} + J_{\text{safety}} + J_{\text{obstacles}} \right) + J_{\text{slack}} \\
\text{s.t.} \quad & \bm{x}_{k+1} = \bm{A}\bm{x}_k + \bm{B}\bm{u}_k, \quad k = 0, \ldots, N-1 \\
& \bm{x}_0 = \bm{x}_{\text{current}} \\
& \|\bm{v}_k\|_2 \leq v_{\max}, \quad k = 0, \ldots, N \\
& \|\bm{a}_k\|_2 \leq a_{\max}, \quad k = 0, \ldots, N-1 \\
& |\dot{\psi}_k| \leq \dot{\psi}_{\max}, \quad k = 0, \ldots, N \\
& |\ddot{\psi}_k| \leq \ddot{\psi}_{\max}, \quad k = 0, \ldots, N-1 \\
& z_{\min} \leq p_{z,k} \leq z_{\max}, \quad k = 0, \ldots, N \\
& \|\bm{p}_k - \bm{o}_i\|_2 \geq d_{\text{safe}} - \epsilon_i, \quad k = 0, \ldots, N, \; i = 1, \ldots, M_k \\
& \epsilon_i \geq 0, \quad i = 1, \ldots, M_k
\end{aligned}
\end{equation}

\section{Reference Trajectory Generation}

The reference trajectory $\{\bm{p}_{\text{ref},k}\}_{k=0}^{N-1}$ is generated using an artificial potential field approach to create smooth, obstacle-aware paths.

\subsection{Attractive Force}

The attractive force toward the goal is:
\begin{equation}
\bm{F}_{\text{attr}} = k_{\text{attr}} \frac{\bm{p}_{\text{goal}} - \bm{p}}{\|\bm{p}_{\text{goal}} - \bm{p}\|}
\end{equation}

Typical value: $k_{\text{attr}} = 1.0$

\subsection{Repulsive Force}

The repulsive force from obstacle $i$ is:
\begin{equation}
\bm{F}_{\text{rep},i} = \begin{cases}
k_{\text{rep}} \left(\frac{1}{d_i} - \frac{1}{d_{\text{inf}}}\right) \frac{1}{d_i^2} \frac{\bm{p} - \bm{o}_i}{d_i} & \text{if } d_i < d_{\text{inf}} \\
\bm{0} & \text{otherwise}
\end{cases}
\end{equation}

where:
\begin{itemize}
    \item $d_i = \|\bm{p} - \bm{o}_i\|$: Distance to obstacle $i$
    \item $k_{\text{rep}} = 3.0$: Repulsion gain
    \item $d_{\text{inf}} = 4.5$ m: Influence radius (= $1.5 \times$ potential\_influence\_dist)
\end{itemize}

The total repulsive force is:
\begin{equation}
\bm{F}_{\text{rep}} = \sum_{i=1}^{M} \bm{F}_{\text{rep},i}
\end{equation}

\subsection{Velocity Computation}

The total force determines the direction of motion:
\begin{equation}
\bm{F}_{\text{total}} = \bm{F}_{\text{attr}} + \bm{F}_{\text{rep}}
\end{equation}

The velocity at each step is computed as:
\begin{equation}
\bm{v}_{\text{ref}} = \frac{\bm{F}_{\text{total}}}{\|\bm{F}_{\text{total}}\|} \cdot v_{\text{desired}}
\end{equation}

where $v_{\text{desired}} = \min(0.6 \cdot v_{\max}, \|\bm{p}_{\text{goal}} - \bm{p}_0\| / T)$

\subsection{Reference Trajectory Integration}

The reference trajectory is generated by forward integration:

\begin{algorithmic}
\STATE $\bm{p}_{\text{ref},0} \gets \bm{p}_{\text{current}}$
\FOR{$k = 0$ \TO $N-1$}
    \STATE Compute $\bm{F}_{\text{attr}}$ and $\bm{F}_{\text{rep}}$
    \STATE Compute $\bm{v}_{\text{ref}}$ using equation (31)
    \STATE $\bm{p}_{\text{ref},k+1} \gets \bm{p}_{\text{ref},k} + \bm{v}_{\text{ref}} \Delta t$
    \STATE Clamp altitude: $p_{z,\text{ref},k+1} \gets \text{clip}(p_{z,\text{ref},k+1}, z_{\min}, z_{\max})$
    \IF{$\|\bm{p}_{\text{ref},k+1} - \bm{p}_{\text{goal}}\| < 0.2$ m}
        \STATE $\bm{p}_{\text{ref},k+1:N} \gets \bm{p}_{\text{goal}}$
        \STATE \textbf{break}
    \ENDIF
\ENDFOR
\end{algorithmic}

\section{Gap-Based Navigation}

The gap navigator identifies navigable corridors in the LiDAR scan and computes intermediate waypoints when the direct path is obstructed.

\subsection{Gap Detection}

A gap is defined as a continuous angular sector where LiDAR readings exceed a threshold:
\begin{equation}
\text{Free}(\theta) = \begin{cases}
\text{true} & \text{if } r(\theta) > 0.85 \cdot r_{\max} \text{ or } r(\theta) = \infty \\
\text{false} & \text{otherwise}
\end{cases}
\end{equation}

Each gap $G_j$ is characterized by:
\begin{itemize}
    \item Start angle: $\theta_{\text{start},j}$
    \item End angle: $\theta_{\text{end},j}$
    \item Center angle: $\theta_{\text{center},j} = (\theta_{\text{start},j} + \theta_{\text{end},j})/2$
    \item Minimum range: $r_{\min,j} = \min\{r(\theta) : \theta \in [\theta_{\text{start},j}, \theta_{\text{end},j}]\}$
    \item Width: $w_j = r_{\min,j} \cdot (\theta_{\text{end},j} - \theta_{\text{start},j})$
\end{itemize}

\subsection{Gap Filtering}

A gap is considered navigable if:
\begin{equation}
w_j \geq w_{\min} \quad \text{and} \quad r_{\min,j} \geq d_{\text{safe}} + d_{\text{margin}}
\end{equation}

Typical values: $w_{\min} = 1.2$ m, $d_{\text{margin}} = 0.2$ m

\subsection{Gap Scoring}

Each navigable gap is scored based on:

\subsubsection{Alignment Score}
\begin{equation}
s_{\text{align},j} = 1 - \frac{|\theta_{\text{center},j} - \theta_{\text{goal}}|}{\theta_{\max}}
\end{equation}

where $\theta_{\text{goal}}$ is the angle to the final goal and $\theta_{\max} = 90°$ (configurable).

\subsubsection{Quality Score}
\begin{equation}
s_{\text{quality},j} = \frac{\theta_{\text{end},j} - \theta_{\text{start},j}}{\pi} \cdot \frac{r_{\min,j} - d_{\text{safe}}}{r_{\max} - d_{\text{safe}}}
\end{equation}

\subsubsection{Combined Score}
\begin{equation}
s_j = \alpha \cdot s_{\text{align},j} + (1-\alpha) \cdot s_{\text{quality},j}
\end{equation}

where $\alpha = 0.5$ (configurable goal\_alignment\_weight).

Additional penalties apply for:
\begin{itemize}
    \item Misalignment: If $|\theta_{\text{center},j} - \theta_{\text{goal}}| > 30°$, multiply score by 0.8
    \item Velocity-based turn severity: If moving fast, penalize gaps requiring sharp turns
\end{itemize}

\subsection{Intermediate Goal Placement}

For the best-scoring gap $G^*$, the intermediate goal is placed at:
\begin{equation}
\bm{p}_{\text{intermediate}} = \bm{p}_{\text{current}} + d_{\text{gap}} \begin{bmatrix}
\cos(\theta_{\text{center}}^*) \\
\sin(\theta_{\text{center}}^*) \\
0
\end{bmatrix}
\end{equation}

where:
\begin{equation}
d_{\text{gap}} = \min(d_{\text{gap,desired}}, r_{\min}^* - d_{\text{safe}} - d_{\text{margin}}, \|\bm{p}_{\text{goal}} - \bm{p}_{\text{current}}\|)
\end{equation}

Typical value: $d_{\text{gap,desired}} = 3.0$ m

\section{Emergency Braking}

If any obstacle is within the emergency radius $d_{\text{emergency}} = 0.4$ m, the controller enters emergency braking mode.

\subsection{Braking Acceleration}

The braking acceleration is computed as:
\begin{equation}
\bm{a}_{\text{brake}} = -\frac{\bm{v}}{v} \cdot 0.5 \cdot a_{\max} + \bm{a}_{\text{escape}}
\end{equation}

\subsection{Escape Direction}

If an escape direction away from obstacles is available:
\begin{equation}
\bm{d}_{\text{escape}} = \frac{\sum_{i \in \mathcal{E}} (\bm{p} - \bm{o}_i)}{\|\sum_{i \in \mathcal{E}} (\bm{p} - \bm{o}_i)\|}
\end{equation}

where $\mathcal{E}$ is the set of obstacles within $d_{\text{emergency}}$.

The lateral escape component is:
\begin{equation}
\bm{a}_{\text{escape}} = 0.3 \cdot a_{\max} \cdot \frac{\bm{d}_{\text{escape}} - (\bm{d}_{\text{escape}} \cdot \hat{\bm{v}})\hat{\bm{v}}}{\|\bm{d}_{\text{escape}} - (\bm{d}_{\text{escape}} \cdot \hat{\bm{v}})\hat{\bm{v}}\|}
\end{equation}

\section{Implementation Details}

\subsection{Optimization Solver}

The MPC problem is solved using CasADi with the IPOPT nonlinear programming solver.

\subsubsection{Solver Settings}
\begin{itemize}
    \item Maximum iterations: 50
    \item Tolerance: $10^{-4}$
    \item Warm starting: Previous solution shifted forward in time
\end{itemize}

\subsection{Obstacle Selection}

At each MPC step, only the most relevant obstacles are included in the optimization:

\begin{enumerate}
    \item Filter obstacles within range: $d_i \leq d_{\text{max}} = 6.0$ m
    \item Score obstacles by: $\text{score}_i = \frac{1}{d_i + 0.1} + \text{forward\_bonus}_i$
    \item Select top $M_{\max} = 15$ obstacles per time step
\end{enumerate}

\subsection{Coordinate Frames}

\begin{itemize}
    \item \textbf{World Frame}: ENU (East-North-Up)
    \item \textbf{PX4 Frame}: NED (North-East-Down)
    \item \textbf{Body Frame}: FLU (Forward-Left-Up)
    \item \textbf{LiDAR Frame}: FLU, scans in XY plane
\end{itemize}

Frame conversions:
\begin{equation}
\begin{bmatrix} x_{\text{ENU}} \\ y_{\text{ENU}} \\ z_{\text{ENU}} \end{bmatrix} = 
\begin{bmatrix} 0 & 1 & 0 \\ 1 & 0 & 0 \\ 0 & 0 & -1 \end{bmatrix}
\begin{bmatrix} x_{\text{NED}} \\ y_{\text{NED}} \\ z_{\text{NED}} \end{bmatrix}
\end{equation}

\section{Parameter Tuning Guidelines}

\subsection{Critical Parameters}

\begin{table}[h]
\centering
\begin{tabular}{@{}lll@{}}
\toprule
\textbf{Parameter} & \textbf{Default} & \textbf{Effect} \\ \midrule
$Q_{\text{pos}}$ & 15.0 & Reference tracking tightness \\
$Q_{\text{goal}}$ & 80.0 & Goal attraction strength \\
$Q_{\text{terminal}}$ & 150.0 & Terminal state importance \\
$R_{\text{acc}}$ & 0.3 & Control smoothness \\
$Q_{\text{rep}}$ & 300.0 & Obstacle repulsion (soft) \\
$W_{\text{obs}}$ & 5000.0 & Obstacle constraint penalty \\
$d_{\text{safe}}$ & 0.8 m & Safety distance \\
$v_{\max}$ & 2.0 m/s & Maximum velocity \\
$a_{\max}$ & 3.0 m/s$^2$ & Maximum acceleration \\
\bottomrule
\end{tabular}
\caption{Key MPC tuning parameters}
\end{table}

\subsection{Tuning Strategy}

\begin{enumerate}
    \item \textbf{Safety First}: Start with large $d_{\text{safe}}$ and high $W_{\text{obs}}$
    \item \textbf{Reference Tracking}: Increase $Q_{\text{pos}}$ for tighter following
    \item \textbf{Smoothness}: Increase $R_{\text{acc}}$ and $R_{\text{jerk}}$ for gentler motion
    \item \textbf{Aggressiveness}: Decrease $d_{\text{safe}}$ and increase velocity limits for faster navigation
    \item \textbf{Gap Navigation}: Adjust $\alpha$ (goal\_alignment\_weight) to balance between goal-seeking and safe gaps
\end{enumerate}

\section{Conclusion}

This MPC implementation provides a robust framework for autonomous drone navigation in cluttered environments. The combination of gap-based navigation, potential field reference generation, and optimization-based control with hard safety constraints enables safe and efficient obstacle avoidance while maintaining smooth flight characteristics.

The key advantages of this approach are:
\begin{itemize}
    \item \textbf{Predictive}: Plans ahead over a 2-second horizon
    \item \textbf{Safe}: Hard constraints prevent collisions
    \item \textbf{Smooth}: Continuous optimization produces natural trajectories
    \item \textbf{Adaptive}: Gap navigation handles complex environments
    \item \textbf{Real-time}: Solves in $<$50 ms for typical scenarios
\end{itemize}

\end{document}
